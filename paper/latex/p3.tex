
% The \section{} command formats and sets the title of this
% section. We'll deal with labels later.
\section*{Problem 3: A Recent Mistake}
\label{sec:p3}

A recent mistake the Mariners organization has made is failing to adequately support and prepare newly-arriving hitters. I believe that the failure of major free-agent acquisitions to transition to playing at T-Mobile Park has played a role in missing the playoffs in each of the last two years.

It is no secret that hitters have had difficulties adjusting to Seattle. Teoscar Hernandez had three successful seasons in Toronto, finishing with a 146, 131, and 128 OPS+, before coming to Seattle and regressing to 108. This year, he joined the Dodgers and promptly regained his old form, slashing .272/.339/.501 and posting a 137 OPS+. Hernandez has also spoken publicly about his inability to feel comfortable hitting at T-Mobile \cite{hereth}.

Unfortunately, Hernandez is  not the only player who has seen a sudden drop in performance after being acquired by the Mariners. Just this year, Jorge Polanco and Mitch Garver experienced significant regression in terms of OPS+ and oWAR, despite being two of Seattle's most expensive off-season acquisitions.

I don't believe that it was an ex-ante mistake to acquire these players. All three were established veterans with a history of positive offensive contributions. Yet, for some reason each failed to live up to the expectations created by their reputations and salaries.

However, I am optimistic that moving forward the organization will be able to address this area. Since the appointment of Dan Wilson as manager and Edgar Martinez as hitting coach, the Mariners ranked second in wRC+, third in runs scored, third in OPS, fourth in batting average, and 12th in strikeout rate \cite{kramer}. A renewed focus on minimizing strikeouts, putting the ball in play, and being aggressive on the basepaths may be key to helping talented hitters succeed at T-Mobile Park.